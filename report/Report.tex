\documentclass[a4paper]{article}
\usepackage{listings}
\usepackage{pgf}
\usepackage[utf8]{inputenc}
\usepackage{verbatim}
\usepackage{titling}
\usepackage{booktabs}
\usepackage{enumitem}
\usepackage{qtree}
\usepackage{amssymb}
\usepackage{amsmath}
\usepackage{times}
\usepackage{dsfont}
\usepackage{titling}
\usepackage[a4paper,
bindingoffset=0.2in,
left=1in,
right=1in,
top=2in,
bottom=1in,
footskip=.25in]{geometry}
\usepackage{cite} %bibtex
\usepackage{pdfpages}

\pretitle{\begin{center}\linespread{1}}
  \posttitle{\end{center}\vspace{0.14cm}} 
\preauthor{\begin{center}\Large}
  \postauthor{\end{center}}

\setlength{\droptitle}{-10em}
\title { \Large{Seminario de Ciencias de la Computaci\'on B}\protect\\
  \large{Heurísticas de Optimización Combinatoria}\protect\\
  \large{Problema del k-\'arbol generador de peso m\'inimo\\con Optimización de Lobos Grises}}


\date{\normalsize{Martes, 25 de Abril, 2023.}}
\author{\normalsize{Profesor: Canek Peláez Valdés}\protect\\
  \normalsize{Autor: Xin Wen Zhang Liu}}\vspace{0.2cm}

\clearpage



\begin{document}
\allowdisplaybreaks
\maketitle

\subsection*{El problema del k \'arbol generador de peso m\'inimo}
La entrada de este problema es una gr\'afica completa no dirigida, sobre la cual se debe encotrar una subconjutno de k v\'ertices los cu\'ales generen un \'arbol de peso m\'inimo en la gr\'afica, entonces el resultado es un conjunto de $k$ v\'ertices y $k-1$ aristas dentro de la gr\'afica. Este problema es NP-duro, con una complejidad polinomial, 


\subsection*{Optimizaci\'on del Lobo Gris / Grey Wolf Optimization}
Esta meta-heur\'istica fue inspirar por el comportamiento depredatorio de los lobos grises, y planteada Mirjalili
~\cite{MIRJALILI201446}. Propuesta como una nueva alternativa a la optimización por enjambre de
part\'iculas, este algoritmo simula el comportamiento depredatorio de los lobos gries , y su comportamiento jer\'arquico dentro de manadas.\\

Cada enjambre de lobos sigue una jerarqu\'ia social, como la que sigue


donde Alpha, Beta y Delta son las 3 mejores soluciones respectivamente, y todas las dem\'as son asignadas a los Omegas.




\section*{Diseño}


\section*{Implementaci\'on}




\section*{Experimentaci\'on y resultados}




\section*{Conclusiones}


\bibliography{citations}{}
\bibliographystyle{plain}
\end{document}